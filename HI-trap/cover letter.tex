%% LyX 2.2.0 created this file.  For more info, see http://www.lyx.org/.
%% Do not edit unless you really know what you are doing.
\documentclass[english]{letter}
\usepackage[T1]{fontenc}
\usepackage[latin9]{inputenc}
\usepackage{babel}
\begin{document}

\letter{To \\
The Associate Editor \\
PRL}

\opening{Dear Editor}

We would like to submit our paper ``Direct verification of the fluctuation-dissipation
relation in viscously coupled oscillators'' for publication in Physical
Review Letters. In this paper, we provide a direct experimental verification
of the fluctuation-dissipation relation in a system of two hydrodynamically
coupled colloidal particles trapped in separate optical tweezers in
a viscous medium (water). In addition, we identify a resonance in
the response function, which is a surprise given the overdamped nature
of the dynamics. 

The fluctuation-response relation is a pillar iof non-equilibrium
statistical physics, relating the intrinsic fluctuations in thermal
equilibrium to the systematic (linear) response to forces applied
externally. The theorem has been widely used in microrheology applications
to determine the characteristics of simple as well as complex, viscoelastic
fluids. However, most research conducted in non-equilibrium statistical
physics tacitly assumes the validity of the fluctuation-dissipation
relation, and there exist few direct experimental verifications of
the result, especially in systems composed of multiple Brownian particles
that, additionally, are coupled hydrodynamically. 

In this work, we use optical tweezers to trap two colloidal particles
in separate traps in a viscous medium, and experimentally demonstrate
that fluctuation-dissipation relation holds. In addition, we observe
that when one of the trapped particles is driven sinusoidally, the
other displays an amplitude resonance at a particular drive frequency.
This observation is facilitated by the fact that the particles are
trapped extremely close to each other, with a surface-surface separation
that is less than the particle radius. This necessarily complicates
the experiment since we need to ensure that there is no cross-talk
in both the trapping and detection schemes, so that the response of
one particle does not spill into that of the other. We check all such
systematics carefully, and finally demonstrate a clear amplitude resonance
in the motion of the driven bead - the first result of this kind in
optical tweezers, where the viscous nature of the trapping medium
damps out all inertial effects, thus precluding the observation of
motional resonances. The width and center frequency of the resonance
can be tuned, so that it can be used as a useful tool for accurate
two-point microrheology. We specifically envisage this approach for
measuring the viscosity of liquids using the trapped colloidal particles
as probes. All our experimental results are also validated by a theoretical
model of the system using appropriate fluid-mediated interactions
represented by the mobility tensors.

We believe that this direct verification of the fluctuation-dissipation
relation will be of interest to the entire readership of the journal,
while the more specific aspects of the resonance and its use in microrheology
will attract the attention of readers working in soft matter physics
and fluid mechanics. We hope you will seek a peer review to determine
the suitability of this manuscript for publication in Physical Review
Letters. 

Yours sincerely,

Ronojoy Adhikari and Ayan Banerjee
\end{document}

%% LyX 2.2.0 created this file.  For more info, see http://www.lyx.org/.
%% Do not edit unless you really know what you are doing.
\documentclass[english,aps, twocolumn, pre,superscriptaddress]{revtex4-1}
\usepackage[T1]{fontenc}
\usepackage[latin9]{inputenc}
\usepackage{array}
\usepackage{booktabs}
\usepackage{multirow}
\usepackage{amsmath}
\usepackage{amssymb}
\usepackage{graphicx}

\makeatletter

%%%%%%%%%%%%%%%%%%%%%%%%%%%%%% LyX specific LaTeX commands.
%% Because html converters don't know tabularnewline
\providecommand{\tabularnewline}{\\}

\@ifundefined{date}{}{\date{}}
%%%%%%%%%%%%%%%%%%%%%%%%%%%%%% User specified LaTeX commands.
\usepackage{hyperref}
\hypersetup{
  colorlinks   = true, %Colours links instead of ugly boxes
  urlcolor     = blue, %Colour for external hyperlinks
  linkcolor    = blue, %Colour of internal links
  citecolor    = blue%Colour of citations
}

\makeatother

\usepackage{babel}
\begin{document}

\title{Fast Bayesian inference of optical trap stiffness and particle diffusion}

\author{Sudipta Bera}

\affiliation{Dept of Physical Sciences, Indian Institute of Science Education
and Research, Kolkata, Mohanpur 741246}

\author{Shuvojit Paul}

\affiliation{Dept of Physical Sciences, Indian Institute of Science Education
and Research, Kolkata, Mohanpur 741246}

\author{Rajesh Singh}

\affiliation{The Institute of Mathematical Sciences-HBNI, CIT Campus, Taramani,
Chennai 600113}

\author{Dipanjan Ghosh}

\affiliation{Dept of Chemical Engineering, Jadavpur University, Kolkata 700032}

\author{Avijit Kundu}

\affiliation{Dept of Physical Sciences, Indian Institute of Science Education
and Research, Kolkata, Mohanpur 741246}

\author{Ayan Banerjee}

\affiliation{Dept of Physical Sciences, Indian Institute of Science Education
and Research, Kolkata, Mohanpur 741246}
\email{ayan@iiserkol.ac.in}

\selectlanguage{english}%

\author{R. Adhikari}

\affiliation{The Institute of Mathematical Sciences-HBNI, CIT Campus, Taramani,
Chennai 600113}
\email{rjoy@imsc.res.in}

\selectlanguage{english}%
\begin{abstract}
Bayesian inference provides a principled way of estimating the parameters
of a stochastic process that is observed discretely in time. The overdamped
Brownian motion of a particle confined in an optical trap is generally
modelled by the Ornstein-Uhlenbeck process and can be observed directly
in experiment. Here we present Bayesian methods for inferring the
parameters of this process, the trap stiffness and the particle diffusion
coefficient, that use exact likelihoods and sufficient statistics
to arrive at simple expressions for the maximum a posteriori estimates.
This obviates the need for Monte Carlo sampling and yields methods
that are both fast and accurate. We apply these to experimental data
and demonstrate their advantage over commonly used non-Bayesian fitting
methods. 
\end{abstract}
\maketitle

\section{Introduction}

Since the seminal contributions of Rayleigh, Einstein, Smoluchowski,
Langevin and others \cite{chandrasekhar1943stochastic}, stochastic
processes have been used to model physical phenomena in which fluctuations
play an essential role. Examples include the Brownian motion of a
particle, the fluctuation of current in a resistor, and the radioactive
decay of subatomic particles \cite{van1992stochastic}. A central
problem is to infer the parameters of the process from partially observed
sample paths, for instance, the diffusion constant from a time series
of positions, or the resistance from a time series of current measurements,
and so on. Bayesian inference provides a principled solution to this
inverse problem \cite{jeffreys1998theory}, making optimal use of
the information contained in the partially observed sample path \cite{zellner1988optimal}. 

The motion of a Brownian particle harmonically trapped in optical
tweezers in a volume of a viscous fluid far away from walls is usually
modelled by the Ornstein-Uhlenbeck stochastic process \cite{van1992stochastic,gardiner1985handbook}.
The stiffness $k$ of the harmonic potential, the friction $\gamma$
of the particle, and the temperature $k_{B}T$ of the fluid are the
three parameters of the stochastic dynamics. For spherical particles
Stokes' law $\gamma=6\pi\eta a$ relates the friction to the particle
radius $a$ and the fluid viscosity $\eta$, while the Einstein relation,
which holds generally, relates the particle diffusion coefficient
$D$ to the temperature and friction through $D=k_{B}T\gamma^{-1}$
\cite{kubo1966fluctuation}. Of these several physical parameters,
any \emph{two} may be chosen independently, and it is conventional
to choose the ratio $k/\gamma$ and $D$ to be independent as they
relate, respectively, to the mean regression rate $\lambda$ and the
volatility $\sigma$ of the Ornstein-Uhlenbeck process (see below). 

Reliable estimation of the stiffness is a necessary first step in
using tweezers for force measurements. An estimation of the friction,
for a particle of known size, provides an indirect measure of the
viscosity of the medium. This microscopic method of viscometry is
of great utility when sample volumes are in the nanoliter range and
conventional viscometric methods cannot be used. Conversely, an estimate
of the friction in a fluid of known viscosity provides a method for
estimating the particle size. In both these cases, an estimate of
the diffusion coefficient provides, by virtue of the Einstein relation,
identical information.

Extant protocols for estimating these parameters from discrete observations
of the position of the Brownian particle can be divided into ``fluctuation''
and ``response'' categories. In the fluctuational methods, the fluctuating
position of the particle is recorded and the known forms of the static
and dynamic correlation functions are fitted to the data. In response
methods, external perturbations are applied to the particle and the
known forms of the average response is fitted to the data. Considerable
care is needed in these fitting procedures to obtain reliable estimates
\cite{berg2004power}. 

In recent work \cite{richly2013calibrating}, Bayesian inference has
been applied to the optical tweezer parameter estimation problem.
The posterior probability distribution of the stiffness and diffusion
coefficient is estimated for a time series of positions, making it
a method of the ``fluctuation'' category. Monte Carlo sampling is
needed to compute the posterior distribution and estimation from a
time series of $10,000$ points requires few tens of seconds. The
advantages of the Bayesian method over conventional calibration methods
have been discussed at length in this work.

In this paper, we present two Bayesian methods, of the fluctuational
category, which do not require Monte Carlo sampling and, consequently,
are extremely fast. For example, they estimate the trap stiffness
and diffusion coefficient from time series containing a million points
in less than a millisecond. The first method extracts information
exploiting the Markov property of the sample path and jointly estimates
the mean regression rate $k/\gamma$ and the diffusion coefficient
$D$. The second method extracts information from the equal-time fluctuations
of the position, which, in equilibrium, cannot depend on the friction
coefficient, and is, then, a function of the stiffness $k$ alone.
In essence, this is a recasting of the ``equipartition'' method
in the language of Bayesian inference.

The first method, in addition to inheriting the generic advantages
of Bayesian inference that have already been pointed out in \cite{richly2013calibrating},
has several specific advantages. First, it uses the exact expression
for the likelihood, which is valid for any $\Delta t$, the interval
at which the position is observed. Therefore, it works reliably with
data acquired at low frequencies. Second, the exact likelihood is
expressed in terms of four sufficient statistics, which are quadratic
functions of the positions. Their use greatly reduces the computation
needed to evaluate the posterior distribution, as four numbers, rather
than a large time series now represents the entire information relevant
to inference. Finally, we are able to obtain exact maximum a posteriori
(MAP) estimates of the mean regression and diffusion coefficients,
and their error bars, in terms of the four sufficient statistics.
This obviates the need for the Monte Carlo sampling or numerical minimization
steps usually required in Bayesian inference. Bayesian credible regions
are easily calculated from the analytically obtained error bars. The
second method is different from the conventional equipartition method
in that it provides a Bayesian error bar, representing a Bayesian
credible interval, rather than a frequentist confidence interval \cite{jaynes1976confidence}.
The combined use of exact likelihoods, sufficient statistics and analytical
MAP estimates yields both speed and accuracy in parameter estimation.

We apply both methods to experimental data and obtain MAP estimates
and error bars that are in excellent agreement with each other. These
estimates are found to be in good agreement with the commonly used
power spectral density calibration method \cite{berg2004power}. The
Bayesian methods of this paper are implemented in a well-documented,
open-source software freely available on GitHub \cite{pybisp}.

The remainder of the paper is organized as follows. In the next section
we recall several key properties of the sample paths and distributions
of the Ornstein-Uhlenbeck process. In Section \ref{sec:Bayesian-inference},
we present the Bayesian methods, in section \ref{sec:Experimental-setup-and}
we describe the experimental setup, and in section \ref{sec:Results}
we apply the Bayesian procedures to the experimental data. We conclude
with a discussion of future directions in the application of Bayesian
inference to optical tweezer experiments and advocate its use as a
complement to standard non-Bayesian methods.

\section{Ornstein-Uhlenbeck process\label{sec:Ornstein-Uhlenbeck-process}}

The Langevin equation for a Brownian particle confined in a potential
$U$ is given by 
\begin{equation}
m\dot{v}+\gamma v+\nabla U=\xi
\end{equation}
where $\xi(t)$ is a zero-mean Gaussian white noise with variance
$\langle\xi(t)\xi(t')\rangle=2k_{B}T\gamma\delta(t-t')$ as required
by the fluctuation-dissipation theorem. In the limit of vanishing
inertia and a harmonic potential, $U=\frac{{1}}{2}kx^{2}$, we obtain
the overdamped Langevin equation 
\begin{equation}
\dot{x}=-\frac{k}{\gamma}x+\sqrt{\frac{2k_{B}T}{\gamma}}\zeta(t),
\end{equation}
where $\zeta(t)$ is now a zero-mean Gaussian white noise with unit
variance. This describes the Ornstein-Uhlenbeck process, whose sample
paths obey the Ito stochastic differential equation 
\begin{equation}
dx=-\lambda xdt+\sigma dW,
\end{equation}
where $\lambda$ is the mean-regression rate, $\sigma$ is the volatility
and $W(t)$ is the Wiener process \cite{gardiner1985handbook}. 

For Brownian motion, the mean regression rate $\lambda=k/\gamma$
is the ratio of the stiffness and the friction while the square of
the volatility $\sigma^{2}=2D$ is twice the diffusion coefficient
$D$. The latter follows by comparing the Langevin and Ito forms of
the path equation and recalling the Einstein relation $D=k_{B}T\gamma^{-1}$
between the diffusion and friction coefficients of a Brownian particle.
In problems involving Brownian motion, it is convenient to work with
the diffusion coefficient, rather than the volatility. 

The ratio of $\lambda$ and $D$ provides the stiffness 
\begin{equation}
\lambda/D=k/k_{B}T
\end{equation}
in units of $k_{B}T$. We note that, in general, there is no relation
between the mean regression rate and volatility of the Ornstein-Uhlenbeck
process and the preceeding identity is a consequence of additional
$physical$ constraints, namely the fluctuation-dissipation and Einstein
relations \cite{kubo1966fluctuation}. 

The transition probability density $P_{1|1}$$(x^{\prime}t^{\prime}|x\thinspace t$),
the probability of a displacement from $x$ at time $t$ to $x^{\prime}$
at time $t^{\prime}$, obeys the Fokker-Planck equation $\partial_{t}P_{1|1}=\mathcal{L}P_{1|1}$,
where the Fokker-Planck operator is 

\begin{equation}
\mathcal{L}=\frac{\partial}{\partial x}\lambda x+\frac{\partial^{2}}{\partial x^{2}}D.
\end{equation}
The solution is a normal distribution, 

\begin{equation}
x^{\prime}t^{\prime}|x\thinspace t\sim\mathcal{N}\left(xe^{-\lambda(t^{\prime}-t)},\frac{D}{\lambda}[1-e^{-2\lambda(t^{\prime}-t)}]\right),
\end{equation}
where $\mathcal{N}(a,b)$ is the univariate normal distribution with
mean $a$ and variance $b$. This solution is exact and holds for
arbitrary values of $|t-t^{\prime}|$. The correlation function decays
exponentially, 
\begin{equation}
C(t-t^{\prime})\equiv\langle x(t)x(t^{\prime})\rangle=\frac{k_{B}T}{k}e^{-\lambda|t-t^{\prime}|}\label{eq:autocorr}
\end{equation}
a property guaranteed by Doob's theorem for any Gauss-Markov processes
\cite{van1992stochastic}. The Fourier transform of the correlation
function gives the power spectral density 
\begin{equation}
C(\omega)\equiv\langle|x(\omega)|^{2}\rangle=\frac{k_{B}T}{k}\frac{1}{\omega^{2}+\lambda^{2}}\label{eq:spectral-density}
\end{equation}
which is Lorentzian in the angular frequency $\omega$. The corner
frequency $f_{c}=\lambda/2\pi$ is proportional to the mean regression
rate. 

The stationary distribution $P_{1}(x)$ obeys the steady state Fokker-Plank
equation $\mathcal{L}P_{1}=0$ and the solution is, again, a normal
distribution,

\begin{equation}
x\sim\mathcal{N}\left(0,\frac{D}{\lambda}\right)=\mathcal{N}\left(0,\frac{k_{B}T}{k}\right).
\end{equation}
Comparing the forms of $P_{1|1}$ and $P_{1}$ it is clear that former
tends to the latter for $|t-t^{\prime}|\rightarrow\infty$, as it
should. 

The transition probability density yields the Bayesian method for
jointly estimating $\lambda$ and $D$ (and hence $k$), while the
stationary distribution yields the Bayesian method for directly estimating
$k$ alone. We now describe these two methods. 

\section{Bayesian inference\label{sec:Bayesian-inference}}

Consider, now, the time series $X\equiv(x_{1},x_{2},\ldots,x_{N})$
consisting of observations of the sample path $x(t)$ at the discrete
times $t=n\Delta$t with $n=1,\ldots,N.$ Then, using the Markov property
of the Ornstein-Uhlenbeck process, the probability of the sample path
is given by \cite{wang1945theory} 
\begin{equation}
P(X|\lambda,D)=\prod_{n=1}^{N-1}P_{1|1}(x_{n+1}|x_{n},\lambda,D)P_{1}(x_{1}|\lambda,D)
\end{equation}
The probability $P(\lambda,D|X)$ of the parameters, given the sample
path, can now be computed using Bayes theorem, as
\[
P(\lambda,D|X)=\frac{P(X|\lambda,D)P(\lambda,D)}{P(X)}
\]
The denominator $P(X)$ is an unimportant normalization, independent
of the parameters, that we henceforth ignore. Since both $k$ and
$\gamma$ must be positive, for stability and positivity of entropy
production respectively, we use informative priors for $\lambda$
and $D$, $P(\lambda,D)=H(\lambda)H(D)$, where $H$ is the Heaviside
step function. This assigns zero probability weight for negative values
of the parameters and equal probability weight for all positive values.
The logarithm of the posterior probability, after using the explicit
forms of $P_{1|1}$ and $P_{1}$, is
\begin{align}
\ln P(\lambda,D|X) & =\frac{N-1}{2}\ln\frac{\lambda}{2\pi DI_{2}}-\frac{\lambda}{2DI_{2}}\sum_{n=1}^{N-1}\Delta_{n}^{2}\nonumber \\
 & +\frac{1}{2}\ln\frac{\lambda}{2\pi D}-\frac{\lambda}{2D}x_{1}^{2}\label{eq:joint-posterior}
\end{align}
where we have defined the two quantities 
\[
I_{2}\equiv1-e^{-2\lambda\Delta t},\quad\Delta_{n}\equiv x_{n+1}-e^{-\lambda\Delta t}x_{n}.
\]

The maximum a posteriori (MAP) estimate $(\lambda^{\star},D^{\star})$
solves the stationary conditions $\partial\ln P(\lambda,D|X)/\partial\lambda=0$
and $\partial\ln P(\lambda,D|X)/\partial D=0$, while the error bars
of this estimate are obtained from the Hessian matrix of second derivatives
evaluated at the maximum \cite{jeffreys1998theory,jaynes2003probability,sivia2006data}.
The analytical solution of the stationary conditions, derived in the
Appendix, yields the MAP estimate to be\begin{subequations}\label{eq:map-estimate}
\begin{eqnarray}
\lambda^{\star} & = & \frac{1}{\Delta t}\ln\frac{\sum x_{n}^{2}}{\sum x_{n+1}x_{n}}\\
D^{\star} & = & \frac{\lambda^{\star}}{N}\left(\frac{\sum\Delta_{n}^{2}}{I_{2}}+x_{1}^{2}\right)\\
\frac{k^{\star}}{k_{B}T} & = & \frac{\lambda^{\star}}{D^{\star}}
\end{eqnarray}
\end{subequations}where both $I_{2}$ and $\Delta_{n}$ are now evaluated
at $\lambda=\lambda^{\star}$ and the sum runs from $n=1,\ldots,N-1$.
These provide direct estimates of the parameters \emph{without} the
need for fitting, minimization, or Monte Carlo sampling. 

The error bars are obtained from a Taylor expansion of the log posterior
to quadratic order about the MAP value,
\begin{equation}
\ln\frac{P(\lambda,D|X)}{P(\lambda^{\star},D^{\star}|X)}\approx-\left(\Delta\lambda,\Delta D\right)^{T}\mathbf{\boldsymbol{\Sigma}}^{-1}\left(\Delta\lambda,\Delta D\right)\label{eq:quadratic-form}
\end{equation}
where $\Delta\lambda=\lambda-\lambda^{\star}$ and $\Delta D=D-D^{\star}$
and $\boldsymbol{\Sigma}^{-1}$ is the matrix of second derivatives
of the log posterior evaluated at the maximum. The elements $\sigma_{\lambda}^{2}$,
$\sigma_{\lambda D}^{2}$, $\sigma_{D}^{2}$ of the covariance matrix
$\boldsymbol{\Sigma}$ are the Bayesian error bars; they determine
the size and shape of the Bayesian credible region around the maximum
\cite{sivia2006data}. Their unwieldy expressions are provided in
the Appendix and are made use of when computing credible regions around
the MAP estimates. We refer to this Bayesian estimation procedure
as ``Bayes I'' below. 

A second Bayesian procedure for directly estimating the trap stiffness
results when $X$ is interpreted not as a time series but as an exchangeable
sequence, or, in physical terms, as repeated independent observations
of the stationary distribution $P_{1}(x)$ \cite{jaynes2003probability}.
In that case, the posterior probability, assuming an informative prior
that constrains $k$ to positive values, is
\begin{equation}
\ln P(k|X)=\frac{N}{2}\ln\frac{k}{2\pi k_{B}T}-\frac{1}{2}\frac{k}{k_{B}T}\sum_{n=1}^{N}x_{n}^{2}\label{eq:k-posterior}
\end{equation}
The MAP estimate and its error bar follow straightforwardly from the
posterior distribution as
\begin{equation}
\frac{k^{\star}}{k_{B}T}=\frac{N}{\sum_{n=1}^{N}x_{n}^{2}},\quad\sigma_{k}=\frac{1}{\sqrt{N}}\frac{k^{\star}}{k_{B}T}\label{eq:k-map}
\end{equation}
and, not unexpectedly, the standard error decreases as the number
of observations increases. This procedure is independent of $\Delta t$
and is equivalent to the equipartition method when the Heaviside prior
is used for $k$. We refer to this procedure as ``Bayes II'' below.

The posterior probabilities in both methods can be written in terms
of four functions of the data
\begin{eqnarray}
T_{1}(X) & = & \sum_{n=1}^{N-1}x_{n+1}^{2},\quad T_{2}(X)=\sum_{n=1}^{N-1}x_{n+1}x_{n},\nonumber \\
T_{3}(X) & = & \sum_{n=1}^{N-1}x_{n}^{2},\qquad T_{4}(X)=x_{1}^{2},
\end{eqnarray}
which, therefore, are the sufficient statistics of the problem. The
\emph{entire} information in the time series $X$ relevant to estimation
is contained in these four statistics \cite{jaynes2003probability}.
Their use reduces computational expense greatly, as only four numbers,
rather than the entire time series, is needed for evaluating the posterior
distributions.

The posterior distributions obtained above are for flat priors. Other
choice of priors are possible. In particular, since both $D$ and
$k$ are scale parameters a non-informative Jeffreys prior is appropriate
\cite{jeffreys1998theory}. Jeffreys has observed, however, that ``An
accurate statement of the prior probability is not necessary in a
pure problem of estimation when the number of observations is large.''
\cite{jeffreys1998theory}. The number of observations are in the
tens of thousands in time series we study here and the posterior is
dominated by the likelihoood rather than the prior. The prior, then,
has an insignificant contribution to the posterior. 

We note that the error bars obtained in both Bayes I and Bayes II
refer to Bayesian credible intervals, which are relevant to the uncertainty
in the parameter estimates, given the data set $X$. In contrast,
conventional error bars refer to frequentist confidence intervals,
which are relevant to the outcomes of hypothetical repetitions of
measurement. In general, Bayesian credible intervals and frequentist
confidence intervals are not identical and should $not$ be compared
as they provide answers to separate questions \cite{jaynes1976confidence}. 

A comparison of the estimates for the trap stiffness obtained from
these independent procedures provides a check on the validity of the
Ornstein-Uhlenbeck process as a data model. Any significant disagreement
between the estimates from the two methods signals a breakdown of
the applicability of the model and the assumptions implicit in it:
overdamped dynamics, constant friction, harmonicity of the potential,
and thermal equilibrium. This completes our description of the Bayesian
procedures for estimating $\lambda$, $D$, and $k$. 
\begin{figure}[t]
\hfill{}\includegraphics[scale=0.3]{figure1}\hfill{}\caption{Schematic of balanced detection scheme to measure Brownian motion
in the $x$ direction from a single trapped polystyrene sphere. Back-scattered
light from the trapped sphere is incident on an edge mirror that divides
it equally between photodiodes PD1 and PD2, having voltage outputs
A and B respectively. The normalized $x$ coordinate of the sphere
at any instant in time is given by $(A-B)/(A+B).$\label{fig:balanced-detection}}
\end{figure}


\section{Experimental setup and data acquisition\label{sec:Experimental-setup-and}}

\begin{figure*}[t]
\includegraphics[scale=0.53]{figure2}

\caption{Discrete sample path and empirical statistics of an optically trapped
Brownian polystyrene bead of radius $a=3\mu$m. Panel (a) shows discrete
observations of one coordinate of the sample path, (b) the histogram
of the position coordinate, (c) the autocorrelation function and (d)
the spectral density. The fits of $\lambda$ from the both the autocorrelation
and the spectral density depend, respectively, on the number of lags
and the number of frequencies used. The guidelines in \cite{berg2004power}
and \cite{tassieri2012microrheology} are followed in obtaining the
fits. \label{fig:discrete-sample-path}}
\end{figure*}
 We collect position fluctuation data of an optically trapped Brownian
particle using a standard optical tweezers setup that is described
in detail in \cite{rsi12}. Here we provide a brief overview. The
optical tweezers system is constructed around a Zeiss inverted microscope
(\emph{Axiovert.A1}) with a $100x$ $1.4$ numerical aperture (NA)
objective lens tightly focusing laser light at $1064$ nm from a semiconductor
laser (\emph{Lasever}, maximum power $500$mW) into the sample. The
back aperture of the objective is slightly overfilled to maximize
the trapping intensity. The sample consists of a dilute suspension
(volume fraction $\phi=0.01$) of polystyrene sphere of diameter $3\mu$m
in 10\% NaCl-water solution, around $20\mu l$ of which is pipetted
on a standard glass cover slip. The total power available at the trapping
plane is around 15 mW. A single particle is trapped at an axial distance
greater than several times its radius to avoid any wall-effects in
the effective drag force due to the water, and it's motion is observed
by back-focal plane interferometry using the back-scattered intensity
of a detection laser at $671$ nm that co-propagates with the trapping
laser. The detection laser power is maintained at much lower levels
than that required to trap a particle. The back-scattered signal from
the trapped particle is measured using a balanced detection scheme,
schematically illustrated in Fig. \ref{fig:balanced-detection}. The
back-scattered light beam is incident on an edge mirror which divides
it equally into two halves that are focused using two lenses of equal
focal length on photodiodes PD1 and PD2 (\emph{Thorlabs} PDA100A-EC
Si-photodiodes of bandwidth 2.4 MHz). The voltage outputs $A$ and
$B$, of PD1 and PD2 respectively, are then combined as $(A-B)/(A+B)$
to give the normalized value of the $x$ coordinate of motion at any
instant of time. The advantage of such balanced detection is that
the intensity fluctuations of the laser are present in both beams
simultaneously and are thus canceled out when the difference is taken.
Note that the direction of the edge mirror decides whether the $x$
or $y$ coordinate of motion is being measured. The mirror is rotated
by $90$ degrees to select between the coordinates. The fast response
of the photodiodes, with a rise time of 50ns at highest gain, ensures
that spurious correlations are kept to a minimum and the data filtering
necessary with slower commercial quadrant photodetectors is avoided
entirely. The data from the photodiodes is logged into a computer
using a National Instruments DAQ system and Labview at sampling rates
between 2-5 kHz. For calibrating the motion, \emph{i.e.} converting
the voltage into physical distance which is necessary for measuring
the diffusion constant, we employ an acousto-optic modulator that
is placed in the back-focal plane of the microscope objective and
scan the trapped bead by distances which are determined from the pixel
calibration of images taken by the camera attached to the microscope
\cite{rsi12}. The balanced detection output is simultaneously measured
to yield the voltage-distance calibration of the detection system.
For the viscosity measurement, we add glycerol to water in fixed proportions
to create 5 samples of different viscosity. The viscosity of each
sample is then measured by a commercial rheometer (Brookfield DB3TLVCJ0)
to match with the experimental results.
\begin{figure}[h]
\includegraphics[width=0.39\textwidth]{pybispPlot}

\caption{Bayesian posterior probability densities. The top panel shows filled
contours of Eq.(\ref{eq:joint-posterior}). The MAP estimate, Eq.(\ref{eq:map-estimate}),
is marked by the filled dot and contours enclosing 70\%, 90\% and
99\% of the probability are labelled. The bottom panel shows Eq. (\ref{eq:k-posterior}).
The MAP estimate, Eq. (\ref{eq:k-map}), is marked by the filled dot
and intervals enclosing 70\%, 90\% and 99\% of the probability are
shaded. The two estimates for $k$ agree to three decimal places.\label{fig:bayes-I-and-II}}
\end{figure}
\begin{figure}[h]
\includegraphics[width=0.44\textwidth]{linefit}

\caption{Variation of trap stiffness $k$ with laser power estimated by the
two Bayesian methods of (Bayes I and Bayes II ) and by the standard
fit to the power spectral density (PSD). The error bars are also shown.
The Bayesian standard error is less than 1 \% of the mean for each
data set.\label{fig:Variation-of-trap}}
\end{figure}

We note that the measured data is a result of a transformation by
the detection apparatus of the physical sample paths. The Bayesian
modelling of the detection apparatus and the transformations it induces
on the physical sample paths is not pursued here. Therefore, we have
a problem of pure estimation and there is no attempt to compare between
alternative models of the data generation process. 

\section{Results\label{sec:Results}}

We now present our results. In Figure. (\ref{fig:discrete-sample-path})
we show a typical sample path of one component of motion in the plane
of the trap, together with its histogram, autocorrelation function
and spectral density. The histogram shows that the distribution of
positions is stationary and very well-approximated by a Gaussian.
The variance $\langle x^{2}\rangle$ is used in the conventional ``equipartition''
method to estimate the spring constant $k$, while the fitting of
the autocorrelation to the exponential in Eq.(\ref{eq:autocorr})
or of the spectral density to the Lorentzian in Eq. (\ref{eq:spectral-density})
is used to estimate the spring constant when the friction constant
is given. For estimation of the stiffness from the PSD, we employ
the procedures suggested in \cite{berg2004power}, including ``blocking''
the data with a bin size of 100 points, and setting the frequency
range for fitting in order to avoid systematic errors due to reliability
issues at both low and high frequencies \cite{berg2004power}. However,
there exist issues in estimating stiffness from both the equipartition
method - where the presence of any additive noise leads to an increase
in the variance that leads to over-estimation of the trap stiffness,
and the PSD - where the standard systematics related to fitting can
be minimized at best but not removed. 

The results of Bayesian inference are shown in Figure. (\ref{fig:discrete-sample-path}).
In panel (a) we show show filled contours of the posterior distribution
in the $\lambda-D$ plane, together with contours of equal probability,
for the ``Bayes I'' method. There is a single maximum at $(\lambda^{\star},D^{\star})$
whose numerically computed value is in excellent agreement with the
analytical MAP estimates of Eq.(\ref{eq:map-estimate}). In panel
(b) we show the Bayesian posterior distribution for the stiffness
for the ``Bayes II'' method. There is remarkably good agreement
between these two Bayesian methods and the power spectral density
method, all three of which are conceptually and procedurally independent,
as shown in Fig. (\ref{fig:Variation-of-trap}) and Table. (\ref{tab:k-versus-power}).
Note that the errors indicated in parenthesis indicate the 1$\sigma$
standard errors in the mean due to measurements over five independent
sets of data. The agreement with the conventional method, shown in
the third column, is within $2-3\%$ in all cases, other than the
first case where the inherent low stiffness of the trap due to low
trapping power led to larger systematics due to the increased influence
of the ambient low frequency noise. The typical length of our time
series is $N\sim30000$ and this gives a Bayesian standard error that
is less than $\frac{1}{2}$\% of the mean. These are well below the
systematic errors and the approximately $3\%$ variability of the
estimates obtained from the fitting procedure.

To compare the Bayesian estimate for the diffusion coefficient we
repeat the experiment for different solvent viscosities keeping both
the laser power ( corresponding to $k\sim$ 6 $pN$) and the particle
size ($3\mu m$) fixed. The Stokes-Einstein relation then provides
an estimate of the diffusion coefficient. We compare this estimate
with the MAP estimate $D^{\star}$ in Table. (\ref{tab:viscometry})
to find agreement to within $10\%$ in all cases. The Stokes-Einstein
relation can be used ``in reverse'' to obtain a MAP estimate of
the viscosity, $\eta^{\star}$, which agrees very well with the known
viscosity of the mixture. The agreement is within 5\% for all cases
with the exception of the last, where the enhanced friction caused
a shift in $\lambda$ towards lower values, once again increasing
the effects of systematics due to ambient low frequency noise sources.
Thus, the Bayesian method can be used for accurate viscometry using
only the discretely observed sample paths of a trapped Brownian particle.
\begin{table}
\begin{tabular}{>{\centering}p{2cm}>{\centering}p{2cm}>{\centering}p{2cm}>{\centering}p{2cm}}
\toprule 
\multirow{2}{2cm}{Laser power (mW) } & \multicolumn{3}{c}{$k$ ($pN\mu m^{-1}$)}\tabularnewline
\cmidrule{2-4} 
 & Bayes I & Bayes II & PSD\tabularnewline
\midrule
10.1 & 1.100(6) & 1.090(6) & 1.20(2)\tabularnewline
\midrule
16.1 & 2.23(1) & 2.230(5) & 2.26(5) \tabularnewline
\midrule
27.2 & 3.88(2) & 3.882(5) &  3.94(6)\tabularnewline
\midrule
31.8 & 4.16(2) & 4.16(2) & 4.22(8)\tabularnewline
\midrule
33.6 & 4.48(2) & 4.48(2) & 4.40(9)\tabularnewline
\midrule
36.8 & 4.83(3) & 4.83(2) & 4.74(10)\tabularnewline
\midrule
43.8 & 6.01(3) & 6.01(3) & 5.98(12)\tabularnewline
\bottomrule
\end{tabular}\caption{Variation of trap stiffness $k$ with laser power estimated by the
two Bayesian methods of this work (Bayes I and Bayes II) and by the
standard fit to the spectral density (PSD) \cite{berg2004power}.
For each value of laser power, the corresponding stiffness is the
mean over 5 different sets of measurements. The variance of mean is
indicated in parentheses. The Bayesian standard error is less than
$1\%$ of the mean for each data set. \label{tab:k-versus-power}}
\end{table}
 
\begin{table}
\begin{tabular}{>{\centering}m{2cm}>{\centering}p{2cm}>{\centering}p{2cm}>{\centering}p{2cm}}
\toprule 
$\eta$ & $D$ & $D^{\star}$ & $\eta^{\star}$\tabularnewline
\midrule 
0.00085 & 1.72 & 1.73(6) & 0.00084(3)\tabularnewline
\midrule 
0.00089 & 1.65 & 1.72(6)  & 0.00085(3)\tabularnewline
\midrule 
0.00137 & 1.07 & 1.05(3) & 0.00139(4)\tabularnewline
\midrule 
0.00197 & 0.743 & 0.732(11) & 0.00200(3)\tabularnewline
\midrule 
0.00243 & 0.603 & 0.586(12) & 0.00250(5)\tabularnewline
\midrule 
0.00487 & 0.301 & 0.276(14) & 0.00530(8)\tabularnewline
\bottomrule
\end{tabular}\caption{Bayesian viscometry in an optical trap. The first column is the viscosity
of the solvent as measured in a rheometer and the second column is
the diffusion coefficient as given by the Stokes-Einstein relation
for that value of the viscosity. The third column is the Bayesian
MAP estimate for the diffusion coefficient and the fourth column is
the value of the viscosity, as given by the Stokes-Einstein relation
for the corresponding value of the diffusion coefficient. There is
a good match between the first and fourth columns. \label{tab:viscometry}}
\end{table}


\section{Discussion\label{sec:Discussion}}

In this work, we have presented an exact Bayesian method for jointly
estimating the mean regression rate and the diffusion coefficient
of an optically trapped Brownian particle. The trap stiffness in temperature
units is obtained as a ratio of the mean regression rate and the diffusion
coefficient. We have also rephrased the standard ``equipartition''
method of directly estimating the trap stiffness as a problem in Bayesian
inference. We have assumed that the Ornstein-Uhlenbeck process is
the data generating model. More general models, which include the
position dependence of the particle friction (as would be the case
in the proximity to walls) or the non-Markovian character of the trajectories
(as would be the case when momentum diffusion is not slow compared
to the time scales of interest) can, with additional effort, be incorporated
in the Bayesian framework. Exact analytical solutions will no longer
be available and one has to resort to approximations of the likelihood,
as short-time expansions of the Fokker-Planck propagator or via numerical
solutions. These introduce additional sources of error which must
be carefully evaluated. In contrast, the method presented here is
exact and can serve as an useful ``null hypothesis'' when comparing
between different models for the data. In future work, we shall present
Bayesian methods for more complex models and provide a fully Bayesian
procedure, embodying Ockham's razor, for the problem of model selection.
\begin{acknowledgments}
RA expresses his gratitude, long overdue, to Professor M. E. Cates
for introducing him to Bayesianism.
\end{acknowledgments}


\subsection{Appendix }

The first partial derivatives of the logarithm of the posterior probability
with respect to $\lambda$ and $D$ are
\begin{eqnarray*}
\frac{\partial\ln P}{\partial\lambda} & = & \frac{N-1}{2}\left(\frac{1}{\lambda}-\frac{I_{2}^{\prime}}{I_{2}}\right)-\frac{\sum\Delta_{n}^{2}}{2DI_{2}}\\
 & - & \frac{\lambda}{2D}\frac{\partial}{\partial\lambda}\left(\frac{\sum\Delta_{n}^{2}}{I_{2}}\right)+\frac{1}{2\lambda}-\frac{x_{1}^{2}}{2D}
\end{eqnarray*}
\begin{eqnarray*}
\frac{\partial\ln P}{\partial D} & = & -\frac{N-1}{2D}+\frac{\lambda\sum\Delta_{n}^{2}}{2D^{2}I_{2}}-\frac{1}{2D}+\frac{\lambda x_{1}^{2}}{2D^{2}},
\end{eqnarray*}
where $I_{2}^{\prime}=2\Delta t\thinspace e^{-2\lambda\Delta t}$.
Setting the second of these equations to zero, $D$ is solved in term
of $\lambda$ and this solution is used in the first equation, together
with the large-sample asymptotics
\begin{alignat*}{1}
\frac{\lambda}{N}\left(\frac{\sum\Delta_{n}^{2}}{I_{2}}+x_{1}^{2}\right) & \thickapprox\frac{\lambda}{\left(N-1\right)}\frac{\sum\Delta_{n}^{2}}{I_{2}},
\end{alignat*}
to cancel all $D$-dependent terms. Setting the resulting equation
to zero and solving for $\lambda$ then yields the MAP estimates in
Eq.(\ref{eq:map-estimate}). 

The second partial derivatives, appearing in Eq. (\ref{eq:quadratic-form}),
are 
\begin{eqnarray*}
\Sigma_{11}^{-1}=\frac{\partial^{2}\ln P}{\partial\lambda^{2}} & = & \frac{N-1}{2}\Big(-\frac{1}{\lambda^{2}}+\frac{I_{2}^{\prime}I_{2}^{\prime}}{I_{2}^{2}}-\frac{I_{2}^{\prime\prime}}{I_{2}}\Big)-\frac{1}{2\lambda^{2}}\\
 & - & \frac{1}{D}\frac{\partial}{\partial\lambda}\left(\frac{\sum\Delta_{n}^{2}}{I_{2}}\right)-\frac{\lambda}{2D}\frac{\partial^{2}}{\partial\lambda^{2}}\left(\frac{\sum\Delta_{n}^{2}}{I_{2}}\right),
\end{eqnarray*}
\begin{eqnarray*}
\Sigma_{12}^{-1}=\frac{\partial^{2}\ln P}{\partial D\partial\lambda} & =\frac{\sum\Delta_{n}^{2}}{2D^{2}I_{2}}+\frac{\lambda}{2D^{2}}\frac{\partial}{\partial\lambda}\left(\frac{\sum\Delta_{n}^{2}}{I_{2}}\right)+\frac{x_{1}^{2}}{2D^{2}},
\end{eqnarray*}
\begin{eqnarray*}
\Sigma_{22}^{-1}=\frac{\partial^{2}\ln P}{\partial D^{2}} & =\frac{N-1}{2D^{2}}-\frac{\lambda}{D^{3}}\Big(\frac{\sum\Delta_{n}^{2}}{I_{2}}-x_{1}^{2}\Big)+\frac{1}{2D^{2}},
\end{eqnarray*}
where $I_{2}^{\prime\prime}=-4\Delta t^{2}\thinspace e^{-2\lambda\Delta t}$.
All the derivatives are evaluated at the maximum given in Eq. (\ref{eq:map-estimate}).
These are assembled into the Hessian matrix $\boldsymbol{\Sigma}^{-1}$and
the matrix is inverted to give the covariance matrix $\boldsymbol{\Sigma}$
in Eq. (\ref{eq:quadratic-form}), whose matrix elements are $\sigma_{\lambda}^{2}$,
$\sigma_{\lambda D}^{2}$, $\sigma_{D}^{2}$ 
\begin{eqnarray*}
\sigma_{\lambda}^{2}= &  & -\frac{1}{\det\boldsymbol{\Sigma}^{-1}}\Sigma_{22}^{-1},\qquad\sigma_{\lambda D}^{2}=\frac{1}{\det\boldsymbol{\Sigma}^{-1}}\Sigma_{12}^{-1},\\
\sigma_{D}^{2}= &  & -\frac{1}{\det\boldsymbol{\Sigma}^{-1}}\Sigma_{11}^{-1},
\end{eqnarray*}
where $\det\boldsymbol{\Sigma}^{-1}=\Sigma_{11}^{-1}\Sigma_{22}^{-1}-\Sigma_{12}^{-1}\Sigma_{21}^{-1}$
and $\Sigma_{21}^{-1}=\Sigma_{12}^{-1}$.

\bibliographystyle{rsc}
\bibliography{exact-inference-of-oup}

\end{document}
